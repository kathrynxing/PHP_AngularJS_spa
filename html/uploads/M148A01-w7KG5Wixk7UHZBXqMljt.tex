\documentclass{article}
\title{M148A01}
\date{2018-01-05}
\author{Kathryn Xing}

%set margin
\addtolength{\oddsidemargin}{-.8in}
\addtolength{\evensidemargin}{-.875in}
\addtolength{\textwidth}{1.75in}

\addtolength{\topmargin}{-.875in}
\addtolength{\textheight}{1.75in}


\usepackage[english]{babel}
\usepackage[utf8]{inputenc}
\usepackage{fancyhdr}
 
\pagestyle{fancy}
\fancyhf{}
\rhead{Xing ID: 20673835}
\lhead{M148 Assigment 1}
\rfoot{Page \thepage}
\usepackage{amsmath}

\begin{document}
\paragraph{Q1: Prove {$x_n$} converges to 0}
\subparagraph{Solution}
Since $a, b > 0$, and $x_0 = a$, it follows that $x_n > 0$ for all n by definition. Thus, ($x_n$) is bounded below by 0.
\subparagraph{}
It is also easy to observe that ($x_n$) is a decreasing sequence.
Let $x^{-1}_n = k$ for some n, we have $k, b > 0$ and 
$ x_{n+1} = \frac{1}{k+b}$
Thus, $x_{n+1} < x_n$ for all n. 
\subparagraph{}
That is, ($x_n$) is a decreasing sequence bounded below, therefore, ($x_n$) converges.
\subparagraph{}
Let $\lim_{n\to\infty} x_n = \lim_{n\to\infty} x_{n+1} = L$, 
\begin{align*}
\lim_{n\to\infty} x_{n+1} &= \lim_{n\to\infty} \frac{1}{x^{-1}_{n} + b}\\
\lim_{n\to\infty} x_{n} &= \frac{1}{\frac{1}{\lim_{n\to\infty} x_n} + b}\\
L &= \frac{1}{\frac{1}{L} + b}\\
L (\frac{1}{L} + b) &= 1\\
1 + Lb &= 1\\
Lb &= 0   (b > 0)\\
Thus, L &= 0\\
\end{align*}

\paragraph{Q2: a. Prove irrationals dense in R}
\subparagraph{Solution}
Select any $p, q \in R$, assume $p < q$ and let $\epsilon = q - p$. Take $n \in N$ such that $\frac{\sqrt{2}}{n} < \epsilon$. 
\subparagraph{}
Thus, between any arbitrary pair of $p, q \in R$ we could construct the irrational ($p + \frac{\sqrt{2}}{n}$) as $p < (p + \frac{\sqrt{2}}{n}) < p + \epsilon $ which is q.

\paragraph{Q2: b. Let $A \subseteq R$ be a dense subset of $R$. Suppose $f, g$ are continuous functions and $f(x) \geq g(x)$ for all $x \in A$. Prove $f(x) \geq g(x)$ for all $x \in R$.}
\subparagraph{Solution}
Assume on the contrary that $f(p) < g(p)$ for some arbitrary $p \in R$.
Since $f, g$ are continuous functions on $R$, $\lim_{x\to p} f(x) = f(p)$ and $\lim_{x\to p} g(x) = g(p)$.
\subparagraph{}
Let $L_{f(p)} = \lim_{x\to p} f(x)$ and $L_{g(p)} = \lim_{x\to p} g(x)$.\
Let $\epsilon = L_{g(p)} - L_{f(p)}$, note that $\frac{1}{2} \epsilon > 0$. There exists $\delta$ such that if $p - x < \delta$, then 
\begin{equation*} 
L_{f(p)} - \frac{1}{2} \epsilon < f(x) < L_{f(p)} + \frac{1}{2} \epsilon
\end{equation*} 
\begin{equation*} 
L_{g(p)} - \frac{1}{2} \epsilon < g(x) < L_{g(p)} - \frac{1}{2} \epsilon
\end{equation*}
Since $A$ is dense on $R$, there exists $a \in A$ such that $p - \delta < a < p$, that is
\begin{equation} \label{eq1}
L_{f(p)} - \frac{1}{2} \epsilon < f(a) < L_{f(p)} + \frac{1}{2} \epsilon
\end{equation} 
\begin{equation} \label{eq2}
L_{g(p)} - \frac{1}{2} \epsilon < g(a) < L_{g(p)} - \frac{1}{2} \epsilon
\end{equation}
It follows that 
\begin{equation} \label{}
\begin{split}
f(a) < L_{f(p)} + \frac{1}{2} \epsilon \leq L_{g(p)} - \frac{1}{2} \epsilon < g(a) \\
f(a) < g(a)
\end{split}
\end{equation}
at which we arrive at a contradiction. Therefore, $f(x) \geq g(x)$ for all $x \in R$.
\paragraph{Q2: c. False}\subparagraph{Solution}
Prove by counter example. Let $f(x) = x^2$ and $g(x) = 0$. Let the set of all irrationals be $A$. It satisfies the condition that on all $x \in A$, $f(x) > g(x)$ however at $x = 0$, $f(x) = g(x)$. 
\paragraph{Q3: a. Suppose $E$ is a bonded subset of $R$ and that $sup E$ is not contained in $E$. Prove that there is a sequence $(x_n)$ converging to $sup E$, with $x_n in E$ and $x_n < x_{n+1}$ for all $n$.}
\subparagraph{Solution}
Since $supE$ is not in $E$, any $e < supE$ for any $e \in E$. We can construct $(x_n)$ by selecting $x_0 = e_0$ with $e_0 < supE$. By definition, $e_0$ is not supreme of E, thus there exist some $e_1 \in E$ such that $e_0 < e_1 < supE$. Select $x_1 = e_1$ and so on, $(x_n)$ can be constructed.
\subparagraph{}
Claim that $(x_n)$ constructed this way satisfies that $x_n < x_{n+1}$ for all $n$ and is bonded above by $supE$. Therefore, $(x_n)$ converges to $supE$ as needed.
\paragraph{Q3: b. show that if a bounded set $E \in Z$, then $supE \in E$.}
\subparagraph{Solution}
Suppose on the contrary that $supE \not\in E$, then for $\epsilon = 0.5$ and $\epsilon = 0.2$ there exists $a, b$ respectively such that:
\begin{equation*}
supE - 0.5 < a < supE
\end{equation*}
\begin{equation*}
supE - 0.2 < b < supE
\end{equation*}
It follows that 
\begin{equation*}
(supE - 0.2) - (supE - 0.5) < b - a
\end{equation*}
\begin{equation*}
0.3 < b - a
\end{equation*}
which contradicts the fact that $a, b$ both belong to $Z$. Therefore, $supE \in E$ for a bounded set $E \in Z$.
\paragraph{Q4: Prove $g$ monotonic}
\subparagraph{Solution} 
Assume on the contrary that $g$ is not monotonic. That is there exist some $p < q < r \in (a, b)$ such that $g(p) < g(q)$ and $g(q) > g(r)$. 
\subparagraph{}
By EVT, in $[p, r]$ for continuous function $f(x)$ there exist some $c_1, c_2 \in [p, r]$ such that for all $x \in [p, r]$
\[f(c_1) \leq f(x) \leq f(c_2) \]
\subparagraph{}
Note that $c_2 \not = p \not = r$ ($c_2$ not end points). Thus, over the open interval (p, r), $f(c_2) \geq f(x)$ for all $x \in (p, q)$
\subparagraph{} 
Thus, for the open interval $(p, r) \in (a, b)$ there is a local maximum. Therefore, a contradiction.
\paragraph{Q5: Prove $\lim _{x\to0^+}{f(x)}$ exists.}
\subparagraph{Solution}
Since $|f'(x)| \leq 5$ for all $x \in (0,1)$, by MVT for any $a, b \in (0,1)$ with $a < b$
\begin{equation*} 
\frac{f(b) - f(a)}{ b - a} \leq 5
\end{equation*}
note that $0<(b-a) < 1$ thus, $f(b) - f(a) < 5$. Since $f(x)$ differentiable on all $(0,1)$, it is continuous on $(0,1)$. That is, $f(x)$ is continuous and bounded on open interval $(0,1)$. 
\subparagraph{}
Let $L = \lim sup f(x)$ to denote $sup f(x)$ on some interval $(0, x_0)$ as $x_0 \to 0$. 
\subparagraph{}
For all $ 0 < \epsilon < 5$, $f(x)$ is continuous and $f'(x) \leq 5$ on $(0, \frac {1}{5} \epsilon)$. Let $sup f(x)$ on $(0, \frac {1}{5} \epsilon)$ exists at some point $a \in (0, \frac {1}{5} \epsilon)$ and $inf f(x)$ at some point $b  \in (0, \frac {1}{5} \epsilon)$. That is
\begin {equation*}
f(a) = sup f(x)
\end{equation*}
\begin {equation*}
f(b) = inf f(x)
\end{equation*}
\subparagraph{}
By MVT, it follows that
\begin {equation*}
|\frac {sup f(x) - inf f(x)}{a - b} |\leq 5
\end {equation*}
\subparagraph{}
and note that 
\[(a - b) < \frac {1}{5} \epsilon\]
\subparagraph{}
Thus, for all $\epsilon > 0$, $\sup f(x)$ and $\inf f(x)$ on $(0, \frac{1}{5} \epsilon)$ satisfy that
\[ \sup f(x) - \inf f(x) < \epsilon\]
that is \[\lim \sup f(x) = lim \inf f(x) = L\]
\subparagraph{}
Therefore, $f(x)$ must converge to some value L as x approaches 0. 
\paragraph{Q6: a. i. Prove $\lim sup_{n\to\infty} x_n$ exists.}
\subparagraph{Solution} 
By definition, it is easy to observe that for the sequence $y_n = sup\{x_n, x_{n+1}, ...\}$, 
\[y_1 \geq y_2 \geq y_3 \geq ...\]
That is, $y_n$ is bounded and is a non-increasing sequence. Therefore it must converge as $n$ goes to $\infty$. Thus, $\lim sup_{n\to\infty}x_n$ must exists.
\paragraph{Q6: a. ii. Find $\lim sup_{n\to\infty} x_n$ in the special case that $x_n = (-1)_n (1+1/n)$.}
\subparagraph{Solution} 
$\lim sup_{n\to\infty} x_n = 1$
\paragraph{Q6: a. iii. Find $\lim inf_{n\to\infty} x_n$ in the special case that $x_n = (-1)_n (1+1/n)$.}
\subparagraph{Solution} 
$\lim inf_{n\to\infty} x_n = -1$
\paragraph{Q6: b. show that for all bounded sequence $(x_n)^\infty_{n=1}$, $\lim inf _{n\to\infty} x_n \leq \lim sup _{n\to\infty} x_n$}
\subparagraph{Solution}
Assume on the contrary that $\lim inf _{n\to\infty} x_n > \lim sup _{n\to\infty} x_n$. Let
\begin{align*}
L &= \lim inf _{n\to\infty} x_n \\
U &= \lim sup_{n\to\infty} x_n\\
\epsilon &= L - U
\end{align*}
\subparagraph{}note that $\epsilon > 0$ by the assumption that $L > U$. \\
\subparagraph{}For $\frac{1}{2} \epsilon$, there exist some N such that if $ n > N$ then, 
\begin {align*}
L - inf x_n &< \frac{1}{2} \epsilon\\
sup x_n - U &< \frac{1}{2} \epsilon
\end{align*}
It follows that 
\begin{equation*}
sup x_n +\frac{1}{2} \epsilon < U < L < inf x_n + \frac{1}{2} \epsilon 
\end{equation*}
\begin{equation*}
sup x_n < inf x_n
\end{equation*}
At which we arrive at a contradiction. Therefore, $U\geq L$ for all bounded sequence $x_n$.
\paragraph {Q6: c. Prove $\lim inf_{n\to\infty} x_n = \lim sup_{n\to\infty} x_n = \lim_{n\to\infty} x_n$.}
\subparagraph{Solution}
Let $L = \lim _{n\to\infty} x_n$. It follows that for any $\epsilon > 0$ there exists some integer $N$ such that if $n>N$ then $|x_n - L| < \epsilon$.
\subparagraph{}
If $\lim inf _{n\to \infty} x_n$ and $\lim sup_{n\to\infty} x_n$ exist, for all $\epsilon > 0$ there is some integer $N$ such that if $n>N$
\begin{equation*}
|inf x_n - L | < \epsilon
\end{equation*}
\begin{equation*}
|sup x_n - L | < \epsilon
\end{equation*}
\subparagraph{}
That is $(inf x_n) $ converges to L and $(sup x_n) $ converges to L as needed.
\end{document}