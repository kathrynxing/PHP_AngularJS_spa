\def\year{2020}\relax

\documentclass[letterpaper]{article} %DO NOT CHANGE THIS
\usepackage{aaai20}  %Required
\usepackage{times}  %Required
\usepackage{helvet}  %Required
\usepackage{courier}  %Required
\usepackage{url}  %Required
\usepackage{graphicx}  %Required
\frenchspacing  %Required
\setlength{\pdfpagewidth}{8.5in}  %Required
\setlength{\pdfpageheight}{11in}  %Required
\setcounter{secnumdepth}{0}  
\usepackage{subfigure}

\begin{document}
% The file aaai.sty is the style file for AAAI Press 
% proceedings, working notes, and technical reports.
%
\title{Your Project Title Here}
\author{Alice Gao, John Smith, Jane Smith\\
\{a23gao, john.smith, jane.smith\}@uwaterloo.ca\\
University of Waterloo\\
Waterloo, ON, Canada\\
}
\maketitle


%%%%%%%%%. Introduction %%%%%%%%%

\section{Introduction}

The {\it Introduction} section describes the background and motivation behind your work.   This section should describe the problem that your project is addressing.  Why is this an important problem to tackle?   Consider using stories, statistics, and facts to really motivate this work.  What are the potential real-world impact, if your project is successful? 

The length of this section is typically 1/2 to 1 page.

%%%%%%%%%. Related Work %%%%%%%%%

\section{Related Work}

Early font generation methodologies concentrated on improving the existing font compositions. Creating fonts based on modifying controlled points on Besier or B-spline curves using minimum spanning tree and convex hull produces novel fonts successfully \cite{devroye1995random}. However this approach were not susceptible to the increasing complexity of style factors and in some cases could not produce fonts that are distinctly different without sacrificing legibility. This problem is solved subsequently by separation of style and content \cite{tenenbaum2000separating,suveeranont2009feature}. Other approaches to model complex logographic languages were modeling characters as a collection of common components; given aesthetic constraints, an optimal solution is generated by searching through a library of components and putting them together with good results (Wang et al. 2008). These example based transformation and interpolation techniques established important groundwork for future research to better understand the mechanism of font creation in various languages.�
Using machine learning techniques to analyze and implement font generation led to many more creative solutions. Neural font style transfer uses CNN (Convolutional Neural Networks) to transfer the style of one image to another (Atarsaikhan et al. 2017). Various types of images can be used to specify font style. It expands the possibilities of novel font generation as it could take as input different textures, or even font from another language. On the other hand Bayesian Program Learning (BPL) is also significant to this field. Its application in generating characters in natural human languages provides interesting insight in language learning as well as font generation (Lake 2015). In general, these techniques were able to accept input of different forms and produce fonts of higher diversity.�
More recently a trend in using GANs (generative adversarial networks) to solve this problem has proven to be fruitful hence is where we are focusing our research. Zi2zi is a method capable of producing multiple fonts with distinct styles at the same time utilizing the power of conditional generative adversarial networks (Kaonashi 2017).�


%%%%%%%%%. Methodology %%%%%%%%%

\section{Methodology}

In one paragraph, describe how you plan to tackle the problem.   What technique(s) do you plan on using?   Which algorithm(s) are you planning to implement? 

Feel free to copy and paste your answer from your topic request here.

%%%%%%%%%. Results %%%%%%%%%

\section{Results}

In one paragraph, describe what you anticipate the result to be. 

Feel free to copy and paste your answer from your topic request here.

%%%%%%%%%. Bibliography %%%%%%%%%
\newpage
\bibliographystyle{aaai}
\bibliography{report}

\end{document}
